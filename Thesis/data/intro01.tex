
\chapter{引言}
\label{chap:intro}

\section{研究背景}
\label{sec:background}
\subsection{大数据研究背景}
随着互联网的快速发展,互联网上的信息呈现了爆发式的增长,互联网已经成为人们获取信
息的一个主要渠道。随着我们可以获得的数据量的不断增加,人们的研究工作也受到了新的
挑战。传统的数据处理手段正愈发显示出其局限性,如何有效对海量的数据进行处理,进而
挖掘出我们所需要的内容逐渐成为一个重要的问题。近几年,“大数据”迅速成为计算机科
学领域非常受关注研究方向。

之前数据挖掘方面许多研究,更多地是关注如何在有限数据的情况下尽可能多地提取出准确
的我们关心的信息。由于受到数据量的限制,很多数据中隐藏的模式和信息并不能被有效地
发现。如今,海量的数据使得数据本身不再是我们研究中的瓶颈,我们关注的重点更多的在
于如何从这些有大量重复冗余的数据中找到我们真正关心的那部分信息。

Doug Laney在\cite{3V}中,提出了“大数据”的3个特点:大的容量(volume)、多样性
(variety)和产生速度(velocity)。在这样一个背景下,本文利用目前所获得的大量的网
页数据,

\subsection{结构化数据}
\label{sec:structuredata}
如果数据可以用明确的结构统一表示
如今互联网上的大部分的文本信息都通过HTML文档的格式传输,HTML是一种半结构化的数据,

\section{相关工作}
\label{sec:relatedwork}

\section{本文重点难点和主要内容}
\label{sec:mainwork}

\section{本章总结}
\label{sec:summaryintro}


%%% Local Variables: 
%%% mode: latex
%%% TeX-master: "../main"
%%% End: 
