
\chapter{书面翻译}
\label{chap:translation}
\begin{center}
       \Large{文档结构相似性算法调研}
\end{center}

\section*{摘要}
  这篇论文对文档结构相似性算法做了简明的调研,包括优化的树编辑距离算法和各种近似
  算法。这些近似算法包括简单加权标签相似度算法,文档结构的傅里叶变换,和将连续序
  列技术应用到结构相似度计算上。我们展示了三个令人惊奇的结果。第一,傅里叶变换的
  方法是所有近似算法中最不精确的一个,同时也是最慢的一个。第二,优化的树编辑距离
  的算法并不一定是最好的用来将不同网站的网页进行聚类的算法。第三,对于许多应用来
  说,最简单的结构的近似可能是最有用也是最有效率的机制。
\section{简介}
随着万维网上大量的文档的出现,自动地处理这些文档,将其应用于信息抽取,近似聚类和
搜索的需求越来越大。在这个领域的主要工作主要集中在文档的内容上。然而,虽然万维网
的继续发展和进化,越来越多的信息被放在结构化的富文本中,从HTML转换到XML。这个结构
化的信息是一个文档意义的重要体现。从文档中辨别出结构上“相似”,或者结构上互
相“包含”的那些文档的方法是一个非常重要那些相关的相似文档关联起来的机制,而这些
文档可能包含不同的文本内容,那些基于文本内容的相似度算法起不到这样的作用。

现在已有几个文档结构在其中起到关键作用的应用。目前的信息提取算法隐式或显式地依赖
文档的结构化元素。结构化的信息能帮助我们将大量的从不同网站上获得的网页整理成一些
可以大致可以比较的集合。这就使得软件能够将可以抽取出正确结果的集合和那些不能抽取
有用信息的集合分离开来。

结构相似度是一个非常重要的话题,有非常多的算法可以计算任意两个文档结构之间的最小
编辑距离。然而,由于这些算法复杂度非常高,通常都需要$O(n^2)$或者更多的时间去计算
距离,因此创造一些更快的,但是距离的计算精确度有些许损失的算法是有可能的。

我们在这篇论文的第二节展示了当前用于检测结构相似度的算法的概述。之后在第三节我们
描述了一个新的基于连续序列的计算结构相似度的近似算法。在第四节我们在速度和精确度
上对比了一下这些近似度算法和优化的树编辑距离算法。在第五节中我们用对不同算法特点
的概括进行了总结。

\section{当前研究状况}
基于树的文档结构性相似度的研究已经有很长的历史了。早期基于树的变换检测来
自[1],[2]。近期Shasha [3], [4], [5]和Chawathe [6],[7]做了一些关于树到树变换的
研究,主要集中在如何创建一个最小的脚本用来进行树之间的转换。在将一些基于结构的相
似度计算修改成半结构的格式方
面也有很多的尝试,包括NiagraCQ [8],Xdiff [9],适用
于XML的Xyleme [10],[11],以及AIDE [12],[13]和适用
于HTML的ChangeDetector\textsuperscript{\texttrademark}。

之前关于HTML文档相似度的工作大部分集中在了内容相似度上,页面分段[15]技术也是一样。
目前的结构相似度集中在了XML的Schema的相似度上。DTD相似度研究集中在对成对的文档和
未知但相似的DTD的近似度计算上。这要求每两篇文档比较一次需要$O(n^2)$的计算复杂度。
其他的工作将文档之间结构相似度的问题转化为用傅里叶变换的时间序列的相似度,实现上
采用快速傅里叶变换以实现$O(n\times lgn)$的复杂度。

在这篇论文中,我们引进了一个将连续序列技术应用在衡量文档结构相似度上的方法。这要
求用$O(k)$的复杂度去创造一个连续序列文档(在这里$k$表示节点的个数),以及提供常数
时间的复杂度进行文档之间的比较。在计算时间上的节省是通过损失计算精确度的得到的,
实际可以任意减小精确度以满足不同的要求。第四节比较了这个技术与其他近似算法在不同
的数据集上的表现。

\subsection{近似算法}
这里我们提供一个对不同类型的用于计算文档相似度算法的概述。我们将描述的衡量标准包
括树编辑距离,标签距离,傅里叶变换和路径相似度。连续序列技术的动机和算法我们将在
第三节给出。

树编辑距离相似度。一些作者提供了一些计算两个树之间优化的树编辑距离的算法。这篇论
文使用Nierman和Jagadish [16]描述的动态编程实现。一般来说,编辑距离衡量的是将一个
树转换为另一个树所要求的插入,删除和更新的最少的节点个数。通过将编辑操作的次数和
较大的那个文档的节点个数之间进行归一化,可以将其转换为相似度的衡量标准。令$N_i$为
文档$D_i$的树形表示的节点集合,于是
\[
TED(D_i,D_j)=\frac{editDistance(D_i,D_j)}{max(|N_i|,|N_j|)}
\]

标签相似度。标签相似度可能是结构相似度最简单的度量方法,因为它只衡量标签集合之间
的
%%% Local Variables: 
%%% mode: latex
%%% TeX-master: "../main"
%%% End: 
