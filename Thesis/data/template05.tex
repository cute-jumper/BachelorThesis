
\chapter{模板生成和内容提取}
\label{chap:template}
\section{模板生成}
\label{sec:templategen}

\subsection{模板介绍和定义}
通过聚类,我们已经得到了一些由同一种模板生成的网页集合,接下来我们将对每个聚类单
独进行处理,生成对应的模板。

模板是实质是这些网页中共有的部分,因此,我们可以期望在大量的由同一模板生成的网页
中,反复大量出现的那些网页结构就是我们需要的模板的一部分。

我们将首先形式化定义模板的组成。模板的基本元素是模板节点(Template Node),每个节
点有两种形式:
\begin{enumerate}
\item 单个不重复的HTML标签,即$<tag>$
\item 由一个或多个HTML标签组成的序列,这些序列可以出现一次或多
  次,即$(\sum_{i=1}^N<tag_i>)+$,其中$N \ge 1$
\end{enumerate}

模板节点的序列可以组成必选节点(Essential Node)或可选节点(Optional Node)。必选
节点对应着由模板节点组成的一个序列,可选节点则对应多个序列,每个序列由不同的模板
节点组成,同时每个序列还对应着一个置信度。我们规定模板中必选节点和可选节点必须间
隔出现,若用$EN_i$表示必选节点,$ON_i$表示可选节点,则一个模板(Template)$Tp$可
以定义为这样一个序列:
\[
Tp=[ON_0]EN_1ON_1EN_2ON_2......EN_n[ON_n]
\]
其中第一个可选节点$ON_0$和最后一个可选节点$ON_n$都不是必需的。
\section{内容提取}
\label{sec:extraction}

\section{本章总结}
\label{sec:summarytemplate}


%%% Local Variables: 
%%% mode: latex
%%% TeX-master: "../main"
%%% End: 
