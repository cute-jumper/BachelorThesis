
\chapter{工作总结和未来展望}
\label{chap:future}
\section{工作总结}
\label{sec:summaryall}
在这篇论文中,我们实现了一个大数据环境下网页模板自动聚类和提取系统。整个系统主要
由三个模块组成:预处理,网页结构相似度计算和模板生成和内容提取,同时这三个模块也
是我们整个系统的重点和难点。此外,还实现了一个用于实验演示的Web Service。

本文的工作可以总结成以下几点:
\begin{enumerate}
\item 设计并实现了一个完整的系统。针对大数据的特点,在每个模块具体实现的时候充分
  考虑了效率问题,做了很多优化;实现了较高程度的自动化处理,最大限度减小人工参与
  的工作量。
\item 高效地实现了后缀树这个数据结构,在此基础上设计了一套适用于在树的先序遍历序
  列中查找重复子树的算法。
\item 实现并改进了最长公共子序列算法,将其用于计算文档的结构相似度;实现了简单的
  凝聚层次聚类算法,并将其用于文档的聚类
\item 实现了一个无监督的模板生成算法,通过人工指定模板中某些部分对应的语义,可以
  使用模板提取新的由该模板生成的网页中对应的信息。
\item 实现了一个Web Service,可以直观地看到网页和模板的匹配情况。
\end{enumerate}
\section{未来工作展望}
\label{sec:futurework}
未来的工作可以针对系统已经实现的主要的三个模块分别展开:
\begin{itemize}
\item 预处理模块:改进利用后缀树查找重复子树的算法,进一步提高算法运行效率。思路
  是改进Ukkonen原有的在线构造算法,在构造树的时候就考虑一些原本的结构信息,尽早过
  滤掉一些不可能的序列。目前后缀树中还有一些信息也可以加入到对重复序列的判断上,
  比如后缀链。
\item 网页结构相似度计算和网页聚类模块:可以进一步改进算法,包括计算相似度的算法
  和聚类算法。在计算最长公共子序列的时候还可以加入一些其他的除了深度之外的更多结
  构信息;可以考虑使用更复杂但是鲁棒性更好的一些聚类算法。
\item 模板生成和内容提取:对模板生成算法做一些修改,比如可以考虑采用半监督的学习
  方法,人工标注一些数据,减小数据中噪音的影响,增加算法鲁棒性;也可以在模板生成
  过程中设计一些更高级的机器学习算法。在内容提取的时候,可以采用树的匹配算法而不
  是序列的匹配算法。
\end{itemize}
%%% Local Variables: 
%%% mode: latex
%%% TeX-master: "../main"
%%% End: 
