% Created 2013-03-08 Fri 17:23
\documentclass[a4paper]{article}
\usepackage[utf8]{inputenc}
\usepackage[T1]{fontenc}
\usepackage{fixltx2e}
\usepackage{graphicx}
\usepackage{longtable}
\usepackage{float}
\usepackage{wrapfig}
\usepackage{soul}
\usepackage{textcomp}
\usepackage{marvosym}
\usepackage[nointegrals]{wasysym}
\usepackage{latexsym}
\usepackage{amssymb}
\usepackage{hyperref}
\tolerance=1000
\usepackage{fontspec}
\usepackage{xunicode}
\usepackage{xltxtra}
\usepackage{xeCJK}
\usepackage{listings}
\usepackage{xcolor}
\usepackage{fancyhdr}
\usepackage{fancybox}
\usepackage{comment}
\usepackage{enumerate}
\usepackage{colortbl}
\usepackage{framed}
\usepackage{amsmath}
\usepackage{algorithm}
\usepackage{algorithmic}

\setmainfont{Times New Roman}
\setmonofont{Courier New}
\setCJKmainfont[BoldFont=YouYuan]{SimSun}
\setCJKfamilyfont{song}{SimSun}
\setCJKfamilyfont{msyh}{微软雅黑}
\setCJKfamilyfont{fs}{FangSong}

\lstset{frame=single}

% new and renew command
\newcommand{\reffig}[1]{Figure~\ref{#1}}
\newcommand{\reftbl}[1]{Table~\ref{#1}}
\renewcommand{\contentsname}{目录}
\renewcommand{\baselinestretch}{1.2}

% In case you need to adjust margins:
\topmargin=-0.0in      %
\evensidemargin=0.5in     %
\oddsidemargin=0.5in      %
\textwidth=5.5in        %
\textheight=8.5in       %
\headsep=0.25in         %

\providecommand{\alert}[1]{\textbf{#1}}

\title{第二周报告}
\author{丘骏鹏}
\date{2013-03-08 Fri}
\hypersetup{
  pdfkeywords={},
  pdfsubject={},
  pdfcreator={Emacs Org-mode version 7.8.11}}

\begin{document}

\maketitle

\setcounter{tocdepth}{3}
\tableofcontents
\vspace*{1cm}

\section{本周进展}
\label{sec-1}

本周主要工作在阅读相关资料和论文。

中文部分:
\begin{itemize}
\item 《XML挖掘:聚类、分类与信息提取》:这本书主要是一些综述,讲解了一些基本概念和
  方法。
\begin{itemize}
\item 关于XML等一系列的概念:DTD,Schema,XSL,XPath,XQuery等
\item 基本的计算XML文档相似度的方法:简单的相似度度量,路径匹配,编辑距离,向量空
    间模型等。不过主要是方法综述,未讲解实现细节。
\item 书中提出的方法:类似路径匹配,加入WordNet进行语义消歧,然后计算出文档之间的
    相似度
\end{itemize}
\item 《基于局部标签树匹配的改进网页聚类算法》:这篇论文主要是将DOM Tree的前几层转化
  为字符串,然后利用字符串的编辑距离衡量DOM Tree的相似度。具体方法为:先将每层的
  标签拼接成字符串,计算相似度,然后不同的层级给予不同的权重,相加得到总的相似度。
  方法比较简单。
\item 《基于网页聚类的web信息自动抽取》:这篇论文主要基于DOM Tree的编辑距离,然后计算
  出平均绝对误差的列相似度,再利用CURE算法进行聚类。CURE聚类算法的论文我看的是
  \href{http://citeseerx.ist.psu.edu/viewdoc/download?doi=10.1.1.83.1599&rep=rep1&type=pdf}{http://citeseerx.ist.psu.edu/viewdoc/download?doi=10.1.1.83.1599\&rep=rep1\&type=pdf}
\end{itemize}

英文部分:
\begin{itemize}
\item UIC Bing Liu的CS511课程课件,讲的是结构化数据抽取中有监督和无监督的wrapper生成
  算法。里面包括了树相似度算法的介绍,有Tree Edit Distance,Simple Tree Matching
  (STM),Center Star method和Particial Tree Alignment。
\item Yanhong Zhai and Bing Liu. Web Data Extraction Based on Partial Tree
  Alignment. 选择性看了关于Partial Tree Alignment的部分。
\end{itemize}

以下两篇是史兴的论文中的参考文献。
\begin{itemize}
\item A. Arasu and H. Garcia-Molina. Extracting structured data from web pages.
  主要讲解的是通过构造Equivalence Class,进行模板的生成。
\item A. Carlson and C. Schafer. Bootstrapping information extraction from
  semi-structured web pages. 首先用partial tree alignment进行对齐,然后利用一些语
  义特征,对这些可能的数据域和schema中的某个column的相似度进行打分,训练模型采用
  bootstrapping。
\end{itemize}

在读的有一篇:TEXT: Automatic Template Extraction from Heterogeneous Web Pages,
文中提出一种效率较高的计算相似度的方法,具体细节部分还在看。您给的那篇论文目前还
没有看。
\section{问题}
\label{sec-2}

Google开源的页面抽取工具我没找到,能否给个链接?
\section{接下来的工作}
\label{sec-3}

\begin{itemize}
\item 这周刚开始看论文,感觉方向有点杂,接下来要做好总结和细化工作。
\item 看一下相关的工具,包括有些论文里提到的工具,看下能否使用以及如何使用,避免重复
  造轮子。
\end{itemize}

\end{document}